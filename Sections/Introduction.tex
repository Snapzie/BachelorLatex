\section{Introduction}
In this thesis we have extended upon a sandbox like demo project from Microsoft \cite{AdventureGame} which resembles an adventure game and is written in a distributed fashion with Orleans. We have then decomposed it into a code repository with the  purpose of using the Combinatory Logic Synthesizer, a type based tool to automatically compose larger systems from a repository of code components, to create different variations of the game based on definitions written in a metalanguage. Our goal has been to synthesize code with timers, synthesizing code where we dispose timers, synthesizing code where we create other actors and to be able to define a list of parameters in our metalanguage and then synthesize code where only the defined parameters are present in the game variation. \todo{Extend introduction to introduce testing?}\\
We will begin by looking at the theory behind the actor model and Orleans, then we will move to the theory of the Combinatory Logic Synthesizer and how it works. After that we will look into more detail about the extensions we have made to the Microsoft project. We will then look into the design decisions about decomposing the code, the issues we faced and what improvements we could have made. We will also look into what makes a good component and ask ourselves if we have achieved good decomposition. Then we will look into the theory of testing software followed by a discussion about our tests \todo{Change for a more comprehensive description}. We will then comment on our results and lastly make our conclusions.
\begin{itemize}
	\item Abstract
	\item Introduction
	\item Actor model - 4
	\item CLS - 4
	\item Our product - 4
	\item Discussion of implementation - 12
	\item Testing theory - 4
	\item Discussion of testing - 12
	\item Results - 4
	\item Conclusion
\end{itemize}