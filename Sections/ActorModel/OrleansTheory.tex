\subsection{Orleans}
Orleans is a framework for C\# developed by Microsoft. The framework simplifies several aspects of the actor model to make it a lot easier for programmers to write actor based code. Some of the simplifications Orleans helps with are lifecycle management of actors in the application code, to deal with distributed race conditions, to handle failures and recovery of actors after server crashes, and to make an abstraction for actor placement.\\
An important abstraction the developers of Orleans have made is the notion of virtual actors. In Orleans we work with virtual actors which are based on the actors described previously, but they implement some important differences which abstract away some of the more cumbersome implementations of actors. The first of these is perpetual existence which states that virtual actors are logical entities which always exists. A virtual actor can not explicitly be created or destroyed. Its existence is unaffected by the server which runs it, and thus also unaffected by the server crashing. Given that a virtual actor always exists, it is also always addressable \cite{OrleansPaper}.\\
Another great help Orleans implement is automatic instantiation. The Orleans' runtime automatically creates instantiations of actors called activations. An actor can only be instantiated if it has a request pending. When a request is sent to an actor the Orleans runtime automatically instantiates an activation on a server. If the server should later crash, the Orleans runtime automatically re-instantiates the actor when the next request to it arrives. This means that the programmer does not have to program actor lifecycles themselves, but can rely on Orleans to automatically figuring out where and when to instantiate an actor. Likewise, the Orleans runtime also automatically disposes unused actors if they have been idle for a long time \cite{OrleansPaper}.\\
Orleans also takes care of the location transparency abstraction. An application running inside an actor, or communicating with an actor does not know its physical location, and neither does it need to, as long as it knows the actors name it can send it a message. Orleans allow programmers to name actors much like one would name an object in an ordinary object-oriented program.\\
Orleans allows for timers which, as the name suggests, specify code that will be run later. A timer is automatically disposed if an actor is recycled by runtime, but can otherwise be specified to fire in even intervals. Because timers 'run in the background' it is important that the programmer makes sure they are disposed when no longer needed.\\
Orleans allows actors to synchronize through RPC-like message passing as discussed in subsection '\nameref{Synchronization}'. This is done by awaiting a response from another actor. Due to atomicity the actor can not progress its method while awaiting a response. Therefore other actors or other tasks are run in the mean time while the actor waits. When the response arrives the actor will then continue the method from where it left off.